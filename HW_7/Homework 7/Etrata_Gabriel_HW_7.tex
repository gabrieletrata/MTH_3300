\documentclass[12pt]{article}
\usepackage{geometry}
\geometry{letterpaper, margin=1in}
\usepackage{setspace}
%\doublespacing
\usepackage{color}
\usepackage{xcolor}
\usepackage{listings}
\usepackage{fancyhdr}
\usepackage[yyyymmdd,hhmmss]{datetime}

% Solarized colour scheme for listings
\definecolor{solarized@base03}{HTML}{002B36}
\definecolor{solarized@base02}{HTML}{073642}
\definecolor{solarized@base01}{HTML}{586e75}
\definecolor{solarized@base00}{HTML}{657b83}
\definecolor{solarized@base0}{HTML}{839496}
\definecolor{solarized@base1}{HTML}{93a1a1}
\definecolor{solarized@base2}{HTML}{EEE8D5}
\definecolor{solarized@base3}{HTML}{FDF6E3}
\definecolor{solarized@yellow}{HTML}{B58900}
\definecolor{solarized@orange}{HTML}{CB4B16}
\definecolor{solarized@red}{HTML}{DC322F}
\definecolor{solarized@magenta}{HTML}{D33682}
\definecolor{solarized@violet}{HTML}{6C71C4}
\definecolor{solarized@blue}{HTML}{268BD2}
\definecolor{solarized@cyan}{HTML}{2AA198}
\definecolor{solarized@green}{HTML}{859900}

% Define C++ syntax highlighting colour scheme
\lstset{language=C++,
	basicstyle=\footnotesize\ttfamily,
	numbers=left,
	numberstyle=\footnotesize,
	tabsize=2,
	breaklines=true,
	escapeinside={@}{@},
	numberstyle=\tiny\color{solarized@base01},
	keywordstyle=\color{solarized@green},
	stringstyle=\color{solarized@cyan}\ttfamily,
	identifierstyle=\color{solarized@blue},
	commentstyle=\color{solarized@base01},
	emphstyle=\color{solarized@red},
	frame=single,
	rulecolor=\color{solarized@base2},
	rulesepcolor=\color{solarized@base2},
	showstringspaces=false
}

\begin{document}
\raggedright{Gabriel Etrata} \hfill \raggedleft{Last Modified on \today\ at \currenttime}\\
\raggedright{MTH 3300}\\%class
\raggedright{Spring 2017}\\
\raggedright{Professor Evan Fink}\\
\textit{Homework 7} \\%topic
\hrulefill\\
\setlength\parindent{24pt} 

Section 5, Problem 18
\begin{lstlisting}
	#include <iostream>
	#include <stdlib.h>
			
	using namespace std;
			
	int main(){
			
			for(int i = 0; i < 1000; i++){
				if (i % 3 == 0){
				cout << "Buzz" << endl;
				} else if (i % 5 == 0){
				cout << "Fizz" << endl;
				} else if (i % 3 == 0 && i % 5 == 0){
				cout << "FizzBuzz" << endl;
				} else {
				cout << i << endl;
				}
			}
			
			return 0;
			}
\end{lstlisting}
			
Section 5, Problem 25
\begin{lstlisting}
	#include <iostream>
	#include <stdlib.h>

	using namespace std;

	int main(){
	
	bool success = false;
	int entry;
	
	while( !sucess ){
		cout << "Enter an integer "Enter an integer (but if it's not between 1 and 100, I'll just ask again): ";
		cin >> entry;
			
			if(entry > 1 && entry < 100){
			sucess = true;
			}	
			
	return 0;
	}
\end{lstlisting}

Section 5, Problem 29
\begin{lstlisting}
	#include <iostream>
	#include <stdlib.h>

	using namespace std;
	
	int main(){
	
	int x = 1;
	while(x < 5){

		for(int i = 1; i <= x; i++){
			cout << i;
			x+=i;
			break;
		}
		cout << ",";
		}
		return 0;
	} //prints out: 1, 1, 1, 1,
\end{lstlisting}

Section 5, Longer Problems 3
\begin{lstlisting}
	#include <iostream>
	#include <stdlib.h>

	using namespace std;

	int main(){
		int nums [9]; //array that holds 10 integers
		cout << "Input 10 integers (separated by spaces, then press Enter): ";
		for(int i = 0; i < 10; i++){
			cin >> nums[i];
		}

		for(int j = 1; j < 10; j++){ //bubble sort algorithm
			for(int k = 0; k < 9; k++){
				if(nums[k] > nums[k+1]){
					int temp;
					temp = nums[k];
					nums[k] = nums[k+1];
					nums[k+1] = temp;
				}
			}
		}

		for(int i = 9; i >+ 0; i--){ //returns the largest integer less than 100
			if(nums[i] < 100 && nums[i] > nums[i-1]){
				cout << "The largest number less than 100 is: " << nums[i] << " ";
				break;
			}
		}

		return 0;
	}
\end{lstlisting}

Section 6, Problem 14
\begin{lstlisting}
	int blah(int &a, int b){
		a++;
		b++;
		return a+b;
	}
	int main(){
		int a = 2, b = 5, c;
		c = blah(a,b);
		cout << a << endl;
		cout << b << endl;
		cout << c << endl;
		return 0;
	}	//prints out:
		//3
		//5
		//9
\end{lstlisting}

Section 6, Problem 18
\begin{lstlisting}
	void replaceMax(int a, int b){
		if(a > b){
			b = a;
		} else if (b > a){
			a = b;
		} 	
	}
\end{lstlisting}

Section 6, Problem 25
\begin{lstlisting}
	#include <iostream>
	
	using namespace std;
	
	int x = 10;
	int fn(int x){

		int y = 2;
		x = y;
		
		return x+y;
	}

	int main(){

		cout << x << endl;
		int x = 5, y = 3;
		y = fn(::x);
		cout << x << y << endl;
		
		return 0;
	}//print outs out:
	 //10
	 //54
\end{lstlisting}

Section 6, Longer Problem 2
\begin{lstlisting}
//Resources used: http://stackoverflow.com/questions/16029324/c-splitting-a-string-into-an-array

	#include <iostream>
	#include <stdlib.h>
	#include <fstream>
	#include <iomanip>
	#include <sstream>

	using namespace std;

	void newWord(string s){
		if(s == "Hillary" || s == "Donald"){
			s = "***";
		}
		cout << s << " ";
	}

	int main(){

	string line;
	int num = 10000;
	string arr[10000]; //an arbitrarily large array (assuming that the largest sentence is 10000 words)
	string word;

	ifstream file("news.txt");
	if(file.is_open() ){
		while(getline (file, line) ){
			stringstream ssin(line);
				for(int i = 0; ssin.good() && i < num; i++){ //splits sentence into individual words
					ssin >> arr[i]; //puts the words in an array, arr
					newWord(arr[i]); //turns the word into "**", if it's "Hillary" or "Donald"
				}
			}
	}
	return 0;
	}

\end{lstlisting}
		

\end{document}