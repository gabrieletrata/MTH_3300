\documentclass[12pt]{article}
\usepackage{geometry}
\geometry{letterpaper, margin=1in}
\usepackage{setspace}
%\doublespacing
\usepackage{color}
\usepackage{xcolor}
\usepackage{listings}
\usepackage{fancyhdr}
\usepackage[yyyymmdd,hhmmss]{datetime}

% Solarized colour scheme for listings
\definecolor{solarized@base03}{HTML}{002B36}
\definecolor{solarized@base02}{HTML}{073642}
\definecolor{solarized@base01}{HTML}{586e75}
\definecolor{solarized@base00}{HTML}{657b83}
\definecolor{solarized@base0}{HTML}{839496}
\definecolor{solarized@base1}{HTML}{93a1a1}
\definecolor{solarized@base2}{HTML}{EEE8D5}
\definecolor{solarized@base3}{HTML}{FDF6E3}
\definecolor{solarized@yellow}{HTML}{B58900}
\definecolor{solarized@orange}{HTML}{CB4B16}
\definecolor{solarized@red}{HTML}{DC322F}
\definecolor{solarized@magenta}{HTML}{D33682}
\definecolor{solarized@violet}{HTML}{6C71C4}
\definecolor{solarized@blue}{HTML}{268BD2}
\definecolor{solarized@cyan}{HTML}{2AA198}
\definecolor{solarized@green}{HTML}{859900}

% Define C++ syntax highlighting colour scheme
\lstset{language=C++,
	basicstyle=\footnotesize\ttfamily,
	numbers=left,
	numberstyle=\footnotesize,
	tabsize=2,
	breaklines=true,
	escapeinside={@}{@},
	numberstyle=\tiny\color{solarized@base01},
	keywordstyle=\color{solarized@green},
	stringstyle=\color{solarized@cyan}\ttfamily,
	identifierstyle=\color{solarized@blue},
	commentstyle=\color{solarized@base01},
	emphstyle=\color{solarized@red},
	frame=single,
	rulecolor=\color{solarized@base2},
	rulesepcolor=\color{solarized@base2},
	showstringspaces=false
}

\begin{document}
\raggedright{Gabriel Etrata} \hfill \raggedleft{Last Modified on \today\ at \currenttime}\\
\raggedright{MTH 3300}\\%class
\raggedright{Spring 2017}\\
\raggedright{Professor Evan Fink}\\
\textit{Homework 4} \\%topic
\hrulefill\\
\setlength\parindent{24pt} 

\begin{enumerate}
	\item \begin{lstlisting}
			#include <iostream>
			#include <stdlib.h>
			#include <fstream>
			
			using namespace std;
			
			int main(){
				int a, b, c;
			
				ifstream dataFile;
				dataFile.open("data.txt");
			
				dataFile >> a;
				dataFile >> b;
				dataFile >> c;
			
				dataFile.close();
				
				return 0;
			}
			\end{lstlisting}
			
	\item \begin{lstlisting}
			#include <iostream>
			#include <stdlib.h>
			#include <fstream>
			
			using namespace std;
			
			int main(){
				string firstWord; secondWord;
			
				ifstream text;
				text.open("mozart.txt");
			
				fileIn >> firstWord;
				fileIn >> secondWord;
				
				cout << secondWord << endl;
			
				text.close();
			
				return 0;	
			}
			\end{lstlisting}
			
	\item \begin{lstlisting}
			#include <iostream>
			#include <stdlib.h>
			#include <fstream>
			#include <iomanip>
			
			using namespace std;
			
			int main(){
				int a;
			
				ifstream numText;
				numText.open("numbers.txt");
			
				fileIn >> a;
			
				ofstream digit;
				digit.open("three.txt");
				digit << setprecision(3) << fixed << a << endl;
				digit.close();
			
				return 0;	
			}
			\end{lstlisting}
			
	\item\begin{lstlisting}
			#include <iostream>
			#include <stdlib.h>
			#include <fstream>
			
			using namespace std;
			
			int main(){
				int a, b, c;
				
				ifstream thatWay;
				thatWay.open("forward.txt");
				
				fileIn >> a;
				fileIn >> b;
				fileIn >> c;
				
				ofstream thisWay;
				thisWay.open("backward.txt");
				
				thisWay << c << b << a << endl;
				
				thatWay.close();
				thisWay.close();
				
				return 0;	
			}
			\end{lstlisting}
		
	\item \begin{lstlisting}
			#include <iostream>
			#include <stdlib.h>
			#include <fstream>
			
			using namespace std;
			
			int main(){
				int a, b, c;
			
				ifstream nums;
				nums.open("numbers.txt");
			
				cin >> a;
				cin >> b;
				cin >> c;

				nums << a + b + c << endl;
			
				nums.close();

				return 0;	
			}
			\end{lstlisting}
	
	\item \begin{lstlisting}
			#include <iostream>
			#include <stdlib.h>
			#include <fstream>
			
			using namespace std;
			
			int main(){
				int a;
			
				ifstream nums;
				nums.open("numbers.txt");
			
				fileIn >> a;
			
				cout << a + 1 << endl;
			
				nums.close();
				
				return 0;
			}
			\end{lstlisting}
			
	\item \begin{lstlisting}
			#include <iostream>
			#include <stdlib.h>
			#include <fstream>
			
			using namespace std;
			
			int main(){
				int a, b, c;
			
				ifstream nums;
				nums.open("numbers.txt");
			
				fileIn >> a;
				fileIn >> b;
				fileIn >> c;
			
				cout << a + b + c << endl;
			
				nums.close();
			
				return 0;	
			}
			\end{lstlisting}
			
	\item \begin{lstlisting}
		#include <iostream>
		#include <stdlib.h>
		#include <fstream>
	
		using namespace std;
	
		int main(){
			string hi = Hello;
			ofstream hello;
			hello.open("text.txt");
		
			hello << hi << endl;
		
			hello.close();
		
			return 0;
		}
			\end{lstlisting}
	
	\item \begin{lstlisting}
		#include <iostream>
		#include <stdlib.h>
		#include <fstream>
		
		using namespace std;
		
		int main(){
			char a, b c;
		
			instream bunchOfChars;
			bunchOfChars.open("in.txt");
		
			fileIn >> a;
			fileIn >> b;
			fileIn >> c;
		
			ofstream firstAndLast;
			firstAndLast.open("out.txt");
			firstAndLast << a << b << c << endl;
		
			bunchOfChars.close();
			firstAndLast.close();
		
			return 0;
		}
			\end{lstlisting}
			
	\item \begin{lstlisting}
		#include <iostream>
		#include <stdlib.h>
		#include <fstream>
		
		using namespace std;
			int main(){
			double p;
		
			ofstream price;
			price.open("price.txt");
		
			price <<  setprecision(2) << fixed << p << endl;
			
			return 0; 
		}
			\end{lstlisting}
	
	\item \begin{lstlisting}
		#include <iostream>
		#include <stdlib.h>

		using namespace std;
		int main(){
			int mystery;
			srand(time(0));
			mystery = rand() %4;
		
			if ()mystery % 4 == 1){
				cout << "Red" << endl;
			} else if (mystery % 4 == 2){
			cout << "Green" << endl;
			} else if (mystery % 4 == 3){
				cout << "Blue" << endl;
			}
			
			return 0;
		}
			\end{lstlisting}
			
	\item \begin{lstlisting}
		#include <iostream>
		#include <stdlib.h>
		
		using namespace std;
		int main(){
			int mystery;
			srand(time(0));
			mystery = rand() % 11;
		
			if(mystery == 1 || mystery == 2){
				cout << "Bad day" << endl;
			} else { 
				cout << "Good day" << endl;
			}
		
			return 0;
		}
			\end{lstlisting}
	
	\item \begin{lstlisting}
		#include <iostream>
		#include <stdlib.h>
	
		using namespace std;
		int main(){
			int mystery;
		
			srand(time()NULL));
			mystery = rand() % 101;
		
			if (mystery < 31){
				cout <<  "precipitation" << endl;
			} else {
				cout << "sun" << endl;
			}
		
			return 0;
		}
			\end{lstlisting}
			
	\item \begin{lstlisting}
		#include <iostream>
		#include <stdlib.h>
		#include <cmath>
		
		using namespace std;
		
		double Pythagorean(int a, int b){
			double c = sqrt(pow(a, 2) + pow(b, 2));
		
		return c;
		}
		
		int main(){
			int a, b;
			srand(time(NULL));
			a = rand() % 10 + 1;
			b = rand() % 10 + 1;
		
			if(Pythagorean(a, b)) > 10){
				cout << "Long" << endl;
			}
		
			return 0;
		}
			\end{lstlisting}
			
	\item \begin{lstlisting}
		#include <iostream>
		#include <stdlib.h>
	
		using namespace std;
		int main(){
			srand(time(NULL));
			int a = rand() % 11;
			int b = rand() % 11;
		
			if( (a + b) % 2 == 0){
				cout << "Go to jail." << endl;
		
			}
			
			return 0;
		}
			\end{lstlisting}
	
	\item \begin{lstlisting}
		#include <iostream>
		#include <stdlib.h>
		#include <fstream>
		
		using namespace std;
		
		int main(){
			string a, b, c;
			int mystery;
			
			srand(time(0));
			mystery = rand() % 101;
			
			instream food;
			food.open("hello.txt");
		
			fileIn >> a;
			fileIn >> b;
			fileIn >> c;
		
			if (mystery > 66){
				cout << a; << endl;
			}
				else if (mystery <= 66 && mystery > 33){
				cout << b; << endl;
			}
				else if (mystery <= 33 && myster >= 0){
				cout << c; << endl;
			}
		
			return 0;
		}
			\end{lstlisting}
			
	\item \begin{lstlisting}
		#include <iostream>
		#include <stdlib.h>
		#include <fstream>
		#include <string>
		
		using namespace std;
		int main(){
			string a, b, c, d;
			
			instream short;
			short.open("short.txt");
			
			fileIn a;
			fileIn b;
			fileIn c;
			fileIn d;
			
			if(a.length() == 6 && b.length() == 6 && c.length() == 6  && d.length() == 6){
				cout << "Not short" << endl;
			} else {
				cout <<< "Short" << endl;
			}
			
			return 0;
		}
			\end{lstlisting}
\end{enumerate}
\end{document}